\documentclass[12pt, letterpaper]{book}

% Packages
\usepackage[utf8]{inputenc}
\usepackage[T1]{fontenc}
\usepackage{lmodern}
\usepackage{geometry}
\usepackage{fancyhdr}
\usepackage{titlesec}
\usepackage{setspace}
\usepackage{parskip}

% Page geometry
\geometry{
    letterpaper,
    margin=1.25in,
    top=1.5in,
    bottom=1.5in
}

% Fix fancyhdr headheight warning
\setlength{\headheight}{14.49998pt}

% Chapter formatting
\titleformat{\chapter}[display]
    {\normalfont\huge\bfseries\centering}
    {}
    {0pt}
    {\Huge}
\titlespacing*{\chapter}{0pt}{-30pt}{40pt}

% Header and footer
\pagestyle{fancy}
\fancyhf{}
\fancyhead[LE,RO]{\thepage}
\fancyhead[RE]{\textit{Core Beliefs}}
\fancyhead[LO]{\textit{\leftmark}}
\renewcommand{\headrulewidth}{0.4pt}

% Line spacing
\onehalfspacing

% Title information
\title{\Huge\textbf{Core Beliefs}}
\author{\Large Faiyaz Chowdhury}
\date{}

\begin{document}

% Title page
\begin{titlepage}
    \centering
    \vspace*{2in}
    {\Huge\bfseries Core Beliefs\par}
    \vspace{1in}
    {\Large Faiyaz Chowdhury\par}
    \vfill
\end{titlepage}

% Table of contents
\tableofcontents
\newpage

% The Core Beliefs

\chapter{Belief 1}

I believe that there isn't a universal core belief or set of core beliefs that are correct. I believe that core beliefs should be optimized to the personality of the bearer and the lifestyle and situation they are regularly faced with. I myself am not religious, but if someone valued their faith, and it helps them get by in life, I would not deem that suboptimal depending on their life circumstances.


\chapter{Belief 2}

Wisdom cannot counteract each other. They can only build on top of each other. Furthermore, that there is no finite set of wisdom etc. Although wisdom can be found in the Bible or Marcus Aurelius's mediations, no one specific anything that contain all that we need. We should seek wisdom throughout our life.


\chapter{Belief 3}

I believe there is good and better. I believe that there is no objective ``bad''. It is a medium for verbal punishment and to many, the punishment is associated to an expectation that is not met. Very often, the expectation is not realistic, and when it is not met, the expectation is the issue, not the person, especially if the person is in a situation in which the expectation is not realistically supposed to be met. By definition, the action that someone takes determines if the action taken is realistic. However, this is not to be conflated with lack for protective action, or even swift retaliation as a deterrent. In prisoner's dilemma, I do believe in being nice, retaliatory, forgiving and clear. But I am not a prisoner. A better path visible to one person may not be visible to the other, and not everyone cares about everyone. This doesn't make them bad, but it does mean that we may not be reaching the better decision by not accounting for the possibility that others may not act in our interest. We want to people in jail because they are likelier to hurt others, not because they are bad. Then if we fix the conditions that make them likelier to hurt others, we can introduce the to society again. People usually assume that because I don't believe in bad that I don't leave myself protected. Animals don't believe in cosmic evil, but they are able to protect themselves. I'm just not self-righteous about it. This absence of ``bad'' is important to me because I feel that the idea of evil can easily be manipulated to make people hurt others with the moral justification that it is less bad to hurt a bad person than a good person, the way Nazis hurt the Jews with the excuse that they were bad. I believe this belief makes it harder to someone else to weaponize me. However, bad may be a simple way to teach children what they should and shouldn't do when their mind is not developed enough to think deeper so that ``bad'' is no longer necessary. I acknowledge that this is a luxury belief, and that if I were in a violent surrounding, I would need to assume the worst quicker. However, I think privileged people overestimate how often people are being malice as an ancestral survivor mechanism. I'm not surrounded by war, maniacs, starving people, or animals that want to kill me. However, I do want to be an inventor, and I do need to be aware that although I have the privilege of not needing to be``bad'', the people using my inventions may not be as privileged, and I need to make sure that people don't use the invention in a way that actually make the world less better. One may argue the existence of one who acts on pure idealistic malice as a hypothetical, but that will always be a pure hypothetical. The truth is there is a reason behind the malice. An animal doesn't need a moral framework to defend itself, and neither do I. I also wanted to add that not only does bad not exist, but perfect doesn't exist either. There is no infinite goodness. To me, perfect is a toxic concept, because it makes us feel that something isn't good enough if it's not perfect, and that something can't be better if it is perfect. To me neither are true. A 3 year old kid can fail a calc test and still be a genius, and a 30 year old can get a perfect score on an 3rd grade math test and still be relatively ignorant. The concept of sub-optimal doesn't exist. Everything tries to be the most optimal it could have been with the information given. We try to do better because we have more information than before. The only time this is not true is in a bizarre hypothetical or incorrect judgement. This can be further proven because when you usually ask people for an example of someone who is bad they will usually tell you someone they never met. Because evil is a fairly tale that exists in a story, even if that story is just history.


\chapter{Belief 4}

I believe that emotion is the very essence of life that gives it meaning, and defines the importance of consciousness.


\chapter{Belief 5}

I am a utilitarian. I focus on long term happiness of everything, integrated from now to the future. I do not assume that sadness is the negation of happiness. So it is good to take self-inflicting risks often.


\chapter{Belief 6}

I believe that reason is a better value than truth. Since objective truth cannot be known, it doesn't help to believe that someone knows objective truth. It is more empathetic, and leaves more avenue for growth if we believed that we thought things because there were useful reasons for believing them, rather than they were true. In a sense, truth isn't binary, and reason is the amount of truth. We believed the Earth was the center of universe because the Earth rotates, which is true, but the sun being center of the solar system is a truer statement. But basic truth doesn't generally allow for such minutia. In the words of Niels Bohr, ``The opposite of a small true statement is a small false statement. The opposite of a big true statement, is a bigger true statement.'' What I think he means is that a small true statement has a reason value of 1, but false 0. But to falsify a statement of reason value 100, one would need a statement of reason value 200. This is also what I think scientists and philosophers mean when they say I know nothing. There is no statement with reason value infinity, besides maybe math which shelters under abstraction, axioms and language. Scientists can't find truth, but they can suggest systems that ``explain'' (are reasons for) more events with more consistency. You may argue against the value of reason over truth, but the truth is we can't know truth, and valuing truth only leads to more blindness. The only people who claim to know truth are bigots and religious followers. True science and philosophy declares ignorance proudly in the name of progress. I'm not saying truth isn't real. I'm saying that it's a direction, not a destination. This in combination with belief 1 and belief 3 can lead to very harmonious discussion where we acknowledge people have different reasons of their values, and arguments become discussions where we all strive to maximize our reason values than damage each others truth as a means of feeling that we have to truth. Reason is additive. Truth is ironically often destructive due to proof by contradiction. Also I want to make it clear that I do not reject reality: I do believe in falsehood, to me this is a statement with 0 reason. If there is no reason for something to happen, then it won't happen. If I design a circuit, electricity won't flow where the metal does not contact. Unless the voltage is so high that it creates an arc, which then becomes a reason. Thus saying that electricity doesn't flow where the metal is not touching is a statement that has a reason value of 0 for most people who don't know about arcs. That is what the would describe as false. For me, there is such a reason. So not false to me. For me, however, the statement like burning my hand will not hurt me has a reason value of 0, but for a doctor, they could think of situations where that may not be always be the case. Just because I don't believe in absolute truth doesn't mean I reject reality or can't make decisions. It just means I seek the most reason. Also, just because I have a reason for something doesn't mean I have to communicate the reason, if there is not reason to communicate it. If I accidentally run over a kid because he was ran in the road at night, I won't tell the grieving mother that I did it because he was running the road. That's extremely cruel. I have no reason to justify myself to her at that moment. Yes to the judge and jury I will, but not to her. Having a reason doesn't mean I have free from retaliation, judgement from others even if I communicate my reasons. People have feelings and given belief 4, my reasons are supposed to based off other's feelings. Before you argue against this belief, stating that it is not objective, you have to realize that people are incapable of being objective no matter how educated they are. There aren't subjective truths and objective truths. Are there are subjective truths and truths that appear objective, and any counter argument you make is a truth that could potentially be disproven by a future physicist. So think before you make a petty counter point that it's not objective. Yeah I know, but no human can be objective. Do not confuse this with lying. I'm saying that when I speak from reason, I'm not lying because I know I can be wrong, and I'm making that clear. When I say something is true, I am 100\% a lie because there will always be a specific condition where that is no longer true.


\chapter{Belief 7}

I believe in outcome independence. There are parts of your brain that want better things. There are parts of your brain that know things it can do to get the better things. When you do things to get the better things, you should judge you action on the basis of your intention, not the result. The result will determine as a datapoint on whether you should do it again, but should not affect your judgement on your intention. Therefore, the result of action is reaction of the environment, and therefore says more about your environment than it does about yourself. This does not contradict Belief 1 because this is a rule for myself, whereas belief 1 is for judging other's beliefs and their life and environment and psychology, which is completely different from mine.


\chapter{Belief 8}

I believe in balance and rhythm in combination with situational optimization. The idea of living everyday like it's your first, and that the current situation has certain gifts and opportunities that are not always available, and that the learning rate is like a logistic function, so it's best to learn and grow in parallel, but prioritizing for whatever the current ``season'' deems ideal.


\chapter{Belief 9}

I believe that reality is best rated and criticized in conjunction with the judgement and opinion of others. A crazy person can think he is doing well, but it is through the communication and requested feedback and we have a symbiotic sense of reality. It is important to note that we should not double count opinions if the opinions are based on the same information. If 20 people think you are bad because you did thing X, then you are bad for doing X, not the fact that 20 people think you did it. Similarly, if 20 people think Y is true because one person told them, then that is only one reason, not 20. Their opinions matter in gauging reality, but we don't want to double count identical references. As a follow up, It matters how we make other people feel, not what they think of us. If I invent something, with everyone thinking I am an idiot, but that device saves everyone, then it is the right action, even if no one knows I did it. When this belief is combined with belief 6, it tackles a lot of the issues of the coherent delusion, because if you have a theory that everyone tells you is insane, and everyone the same exact reason, then it is likely that the theory is flawed. That is the exact reason I asked you for your opinion in the first place.


\chapter{Belief 10}

I distinguish between Material Scarcity (food, shelter, resources) and Abstract Scarcity (status, prestige, ``being the best''). Material Scarcity: We must navigate the economy to secure our needs, but technology continuously reduces this scarcity (positive sum). Abstract Scarcity: Values like ``power,'' ``fame,'' or ``ranking'' are inherently Rivalrous (Zero-Sum). For me to be \#1, you must be \#2. This creates misery even amidst abundance. Therefore, I reject competition over rivalrous abstract values. Instead, I strive for Non-Rivalrous Meritocracy: subjective mastery where my success does not require another's failure. I seek ``Surplus Values'' like love, creativity, discovery, and community. In a globalized world of 8 billion people, attempting to be statistically ``The Best'' is a recipe for neurosis. I choose instead to be ``The Only''---optimizing for a unique intersection of skills and values (my personal monopoly) rather than competing on a crowded ladder.


\chapter{Belief 11}

I am a compatibilist. Meaning that I believe the world is deterministic. Even our action choices are based of our past experiences and genetics. As a consequence moral judgement is meaningless because none of us chose to be raised or exposed to the values that we have. However, we still have choices. It's just important to keep I mind the choices themselves are functions of prior experiences.


\chapter{Belief 12}

I believe that because we get used to things, over a larger period of time, the ``good'' experiences cancel out with the ``bad'' experiences. I call it law of conservation of emotion. However, this isn't a strict causal law, but more like an empirical law, like Moore's law. It isn't always true. It just happens to be true very often. In psychology, called the opponent process theory. Like if we struggle through exercise then we feel good afterwards. However if we do drugs, then we suffer later due to the craving. I think it makes acute stress less of an issue and less worrysome.


\chapter{Belief 13}

Combining values 3,4,5 with 12 is a strange concept. Although opponent process theory is a thing, the allostatic load implies that there is a slight potential increase or decrease in the baseline over time through chronic joy or chronic stress. Eudaimonic satisfaction is thus the goal, by striving for a healthy balanced life and helping others reach the same. Also to help is generally saving the best for last, so that there is always something to look forward to by front loading the pain, and backloading the pleasure.


\chapter{Belief 14}

I believe in the power of curiosity and inquisitivity. The power is not in the answer, but the act of finding a valuable question, and trying to find valuable answers. Through reasons, and our effort in discovering them, our beliefs become closer to core beliefs. They before more natural to act on and easier to recall. They become easier to communicate. Easier to communicate because, through recursively distilling reasons, I hypothesize that they become more universal, and thus easier for others to understand. For example, people may not understand why I dance so much. But if I told them I do it because I like it, and it keeps me fit and mentally and socially stimulated, and I like listening to music, and I have chances to meet a romantically partner, all these reasons are more relatable, and the further I go deeper in the reason, the more finite and deterministic the answer becomes. I believe the power of asking why largely from the hope that in the analogy of a tree, all leaves get closer together the closer you go to the root. Not only have you created a path from the leaf to the root, thus understanding you own actions, but you can explain their value through a common node, be that the root or the closest common branch. However, this thought requires time and the time to think is a luxury that may not always be available. I acknowledge my privilege, and not everyone has the time to find a deeper why, and that others have gone even deeper than myself. Furthermore, in a more homogenistic society, the common node comes is closer to the top. Thus, as humans, to understand universality of reason, we to look not only at other humans and other cultures and governments, but other animals and cycles of nature, and physical systems. Also I want to note that I do see value in the branches. They are cached secondary and tertiary values. They are important so that one does not spend a bizarre amount to time to make the simplest of decisions. However, I do see value in the exercise of re-discovering the path from leaf to node, to make the action more confident, automatic and genuine. I spend the last 5 hours discussing my beliefs with you. This is part of conversation is part of this very process. I understand that it's slow, but the planning makes it faster when needed. Like Eisenhower says, ``Plans are useless, but planning (making a decision tree) is everything.'' In a way, ``Philosophies useless, but philosophizing (making a tree of values) is everything.'' I believe that this root node is what I describe in belief 4 and 5, just human happiness. And I don't need to go any further because that would be recursion. I like happiness as a root node because it explains all conscious action. If someone values truth, they do it because it gives them happiness. If they value sex, that explains their action too. But truth doesn't explain why someone chose to order a burger instead of a sandwich, but happiness does. In fact it also explains how animals and AI works, with the backward propagation.


\chapter{Belief 15}

I believe that virtue and idealism are different. Virtue is doing the actions that yields the better outcome regardless of everyone else's actions. But idealism is the action that yields the best outcome given that all or most people believe the same. I don't believe that I have the right to control others nor push my expectations or values onto them. So I believe in virtue over idealism under these definitions.


\chapter{Belief 16}

I acknowledge that being selfish is normal and healthy, as it is needed for survival, unless someone is taking care of you, but projecting that expectation is idealism, not a virtue. But I believe that true kindness is having qualities that allow us to be selfish whilst benefitting others. To me, this is empathy (being happy from other's happiness), intelligence (finding ways to make both parties happy), willpower (being able to perform this action, even when difficult), fulfillment (working towards being healthy and satisfied where kind actions don't always have to feel like a sacrifice but instead a pleasure), persistent conscientiousness (making sure that the path to being healthy doesn't do so at the sacrifice of others. The ends don't necessarily justify the means, especially if the means are more costly than the ends) and morals (cached learnings and ingrained habits so that beneficial actions are more likely to be automatic). Many strive to be less selfish. But just beating yourself over it only makes things worse. If we raise children to be empathetic (by rewarding in the presence of others, eating together), intelligence (reward learning and problem solving), willpower (limit easy dopamine access like porn and social media and video games), fulfilled (making sure the child has their needs met), conscientious (the child is raised to think about all their actions) and philosophical (trained to access the moral value of their own values) then being kind doesn't have to be a goal. It's automatic. Kindness aligns with increasing happiness debt too because you go out of your way for someone, causing some pain, but the relationship will make future experiences better. Combining with belief 14, and caching values like ``don't hurt people'' as branches of the philosophy tree, and making it a habit. One can learn to be kind even in difficult scenarios. I make this rule because I know so many people who value kindness but are just not kind themselves. It is because they fail to take care of themselves to the point where they are able to be kind.


\chapter{Belief 17}

I believe that expectation and comparison is a cause of a lot of misery. Expectation is useful because it allows us to make decisions that allow us to pick the better option. But in order to do so, we need to make an expectation of both the options. Once the decision is make however, we need to try to forget the expectation, because if the reward is less than the expected value, we feel misery. Arguably that misery is a learning feedback, and we do need to learn from the result. But we apply expectations in so many more harmful ways, with false promises, glorifying future rewards, pressure to withhold unsustainable standards.


\chapter{Belief 18}

I think identity is an expectation we put on ourselves and others for simply being something. Often something we can't change, like race, and often something we believe. When I identify as Faiyaz, I think, ``I'm Faiyaz. I'm supposed to be kind.'' And act in accordance. But I don't believe we need an expectation to perform if we ``do the homework'' of building our values. Then it just becomes a burden on ourselves and a finger to point at others. There is also the preference to dislike people who identify differently, but I should be the actions and not the identity that ought to determine this in my opinions. A Muslim Indian doctor has more in common with a Jewish American doctor than a Jewish American doctor does to a Jewish American soldier. Also, due to abstract nature of many ideals, like religion and politics, people have the illusion that they have the same identity when they don't, or that they don't have the same identify when they do. So it is important that we are very aware of our identities and how that may cause us to hurt others.


\chapter{Belief 19}

As a means of appreciating life via understanding the mind, I think it is important to propagate correlations of positive experiences, and isolate the correlations of bad experiences if it doesn't compromise decision making negatively. For example if you meet someone new, and they remind you of your best friend, it's okay to want to hang out with them if it becomes a social lubricant, if that person is good. But if you meet someone new, and they look like someone you hate, then it is wise to say to yourself, ``oh I only feel like hating them because they look that that jerk. I should have an open mind about them.''


\end{document}

